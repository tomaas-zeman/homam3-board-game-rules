% !TeX spellcheck = en_US
\documentclass[12pt]{article}

% Language setting
\usepackage[T1]{fontenc}
\usepackage[spanish]{babel}

\addto\captionsspanish{
  \renewcommand{\contentsname}
    {Contenido}
}

\def\sections{sections/translated/es}
\def\lang_header_adjustment{-6px}
\newcommand\subsectionspacing{
  \titlespacing*{\subsection}{0em}{0.9em}{0.2em}
}

\newcommand{\versionlabel}{Versión}
\newcommand{\versionwarning}{Se trata de una versión inédita creada en GitHub a partir del commit}

\newcommand{\intro}{
  Se trata de un proyecto impulsado por la comunidad, que cuenta con un \href{\repourl}{repositorio en GitHub}.
  Todo el mundo es bienvenido a contribuir, hacer cambios y corregir errores.
  Si simplemente quieres dejar comentarios, por favor hazlo en el \href{\bggthreadurl}{hilo original de BoardGameGeek}.
}

\newcommand{\heegusquote}{
  \textit{Lo hago porque me apasiona aprender sobre los juegos y entender sus complejidades.}
  \\
  {\small — Heegu}
}

\newcommand{\pageshorthand}{p.}

\newcommand{\qrgithub}{Escanear para abrir el repositorio \textbf{GitHub}.}
\newcommand{\qrbgg}{Escanear para abrir el hilo \textbf{BoardGameGeek}.}

\title{\includegraphics[width=6cm]{\images/title.png}\\Reescritura}

% Set page size and margins
\usepackage[
  a4paper,
  top=2cm,
  bottom=3cm,
  left=2cm,
  right=2cm,
  marginparwidth=1.75cm,
  footskip=2.05cm
]{geometry}

% Useful packages
\usepackage[export]{adjustbox}
\usepackage{amsmath}
\usepackage{array}
\usepackage{caption}
\usepackage[strict]{changepage}
\usepackage{enumitem}
\usepackage{etoolbox}
\usepackage{float}
\usepackage{fullwidth}
\usepackage{graphicx, trimclip}
\usepackage[colorlinks=true, allcolors=blue]{hyperref}
\usepackage{hyperref}
\usepackage[noautomatic, nonewpage]{imakeidx}
\usepackage{multicol}
\usepackage[super]{nth}
\usepackage{outlines}
\usepackage{paracol}
\usepackage[section]{placeins}
\usepackage{setspace}
\usepackage{stfloats}
\usepackage{subcaption}
\usepackage[usetransparent=false]{svg}
\usepackage{tabularx}
\usepackage[subfigure]{tocloft}
\usepackage{tikz}
\usepackage{titlesec}
\usepackage{verbatim}
\usepackage{varwidth}
\usepackage{wrapfig}
\usepackage[most]{tcolorbox}
\newtcolorbox{scaledfigure}[1][]{height fill, space to=\myspace,#1}
\hypersetup{
  colorlinks=true,
  linkcolor=goldenbrown,
  filecolor=magenta,
  urlcolor=cyan,
  pdftitle={Heroes of Might \& Magic III Rule Book},
  pdfpagemode=UseNone,
}
% Set the default spacing between paragraphs. Remove indentation.
\usepackage[skip=6pt, indent=0pt]{parskip}
\setstretch{1}

% Add dots to the table of contents
\renewcommand{\cftsecleader}{\cftdotfill{\cftsecdotsep}}
\renewcommand\cftsecdotsep{\cftdot}
\renewcommand\cftsubsecdotsep{\cftdot}

\captionsetup[figure]{labelformat=empty}
\captionsetup[subfigure]{labelformat=empty, singlelinecheck=off, justification=centering}
\usetikzlibrary{shadows, shadows.blur, calc, backgrounds}

\setlength{\columnsep}{1cm}
\newtoggle{printable}

% Variables
\def\_assets{assets}

\def\art{\_assets/art}
\def\cards{\_assets/cards}
\def\examples{\_assets/examples}
\def\images{\_assets/images}
\def\layout{\_assets/layout}
\def\map_locations{\_assets/map-locations}
\def\skills{\_assets/skills}
\def\spells{\_assets/spells}
\def\svgs{\_assets/glyphs}
\def\notes_svgs{\svgs/for-notes}
\def\tables{\_assets/tables}
\def\qr{\_assets/qr-codes}

\renewcommand{\labelitemi}{\includegraphics[width=0.7em, valign=c]{\layout/listdot.png}}

% Colors
\definecolor{amber}{rgb}{1.0, 0.49, 0.0}
\definecolor{antiquewhite}{rgb}{0.98, 0.92, 0.84}
\definecolor{arylideyellow}{rgb}{0.96, 0.89, 0.58}
\definecolor{cadmiumgreen}{rgb}{0.0, 0.42, 0.24}
\definecolor{darkcandyapplered}{rgb}{0.64, 0.0, 0.0}
\definecolor{goldenbrown}{rgb}{0.6, 0.4, 0.08}

% Command to frame images
\newcommand\framedimage[2][]{%
  \begin{tikzpicture}
    \draw (0, 0) node[inner sep=0] {\makebox[#1][c]{\includegraphics[width=#1]{#2}}};
    \draw [bordermidyellow, thick] ([xshift=+1pt, yshift=-1pt] current bounding box.north west) rectangle ([xshift=-1pt, yshift=1pt] current bounding box.south east);
    \draw [borderoutyellow, thick] (current bounding box.north west) rectangle (current bounding box.south east);
    \draw [borderinyellow, thick] ([xshift=+3pt, yshift=-3pt] current bounding box.north west) rectangle ([xshift=-3pt, yshift=3pt] current bounding box.south east);
  \end{tikzpicture}}
% End of drop frame definition

\titleformat{\section}
{\huge}
{\filright
\footnotesize
\enspace SECTION \thesection\enspace}
{8pt}
{\Huge\bfseries\filcenter\uppercase}
%Create section heading with graphics. Argument one is heading name, argument two is picture to use on the left.
\providecommand{\sectionheadertext}[1]{
  \fontfamily{ptm}\selectfont{
    \color{antiquewhite} \section*{\MakeUppercase{#1}}
  }
}
\newcommand{\addsection}[2]{
  \vspace*{-5em}
  \hspace*{-1em}
  \makebox[0pt][l]{
  \raisebox{-\totalheight}[0pt][7pt]{
    \begin{tikzpicture}
      \draw (0, 0) node[inner sep=0] {\includegraphics[width=\linewidth, height=0.2\linewidth]{\layout/section_heading.jpg}};
      \draw (-6.2, 0) node {\includegraphics[width=0.125\textwidth]{#2}};
    \end{tikzpicture}
    }
  }
  \begin{fullwidth}[leftmargin=0.16\textwidth]
    \begin{center}
      \vspace*{\lang_header_adjustment}
      \sectionheadertext{#1}
      \cleardoublepage\phantomsection\addcontentsline{toc}{section}{\protect\numberline{}#1}
    \end{center}
  \end{fullwidth}
  \vspace{1.75em}
}
%End of create section heading.

\newcommand\picdims[4][]{%
  \setbox0=\hbox{\includegraphics[#1]{#4}}%
  \clipbox{.5\dimexpr\wd0-#2\relax{} %
    .5\dimexpr\ht0-#3\relax{} %
    .5\dimexpr\wd0-#2\relax{} %
    .5\dimexpr\ht0-#3\relax}{\includegraphics[#1]{#4}}}

\tikzset{
  thick/.style=      {line width=1.3pt},
  very thick/.style= {line width=1.7pt},
  ultra thick/.style={line width=2.2pt}
}

\definecolor{borderoutyellow}{HTML}{DBCA86}
\definecolor{borderinyellow}{HTML}{B09E69}
\definecolor{bordermidyellow}{HTML}{6f6749}
% Create note box
\providecommand{\notefont}[0]{\fontfamily{ptm}\selectfont}
\newcommand{\note}[2]{
  \begin{tikzpicture}
    \draw (0, 0) node[inner sep=0] {\makebox[\linewidth][c]{\picdims[width=\linewidth]{\linewidth}{#1\baselineskip}{\layout/table-background.jpg}}};
    \draw [borderoutyellow, very thick] (current bounding box.north west) rectangle (current bounding box.south east);
    \draw [borderinyellow, thick] ([xshift=+2.8pt, yshift=-2.8pt] current bounding box.north west) rectangle ([xshift=-2.8pt, yshift=2.8pt] current bounding box.south east);
    \node at (current bounding box.center) {
      \begin{varwidth}{0.85\linewidth}
      \notefont{
        \color{arylideyellow}
        \hypersetup{linkcolor=amber}
        #2
        \hypersetup{linkcolor=goldenbrown}
      }
      \end{varwidth}
    };
    \begin{pgfonlayer}{background}
      \draw [shade, blur shadow={shadow opacity=15}] (current bounding box.north west) rectangle (current bounding box.south east);
    \end{pgfonlayer}
  \end{tikzpicture}
}

\newcommand{\hommtable}[2]{
  \begin{tikzpicture}
    \draw (0, 0) node[inner sep=0] {\makebox[\linewidth][c]{\picdims[width=\linewidth]{\linewidth}{#1\baselineskip}{\layout/table-background.jpg}}};
    \draw [bordermidyellow, thick] ([xshift=+1pt, yshift=-1pt] current bounding box.north west) rectangle ([xshift=-1pt, yshift=1pt] current bounding box.south east);
    \draw [borderoutyellow, thick] (current bounding box.north west) rectangle (current bounding box.south east);
    \draw [borderinyellow, thick] ([xshift=+3pt, yshift=-3pt] current bounding box.north west) rectangle ([xshift=-3pt, yshift=3pt] current bounding box.south east);
    \node at (current bounding box.center) {
      \begin{varwidth}{0.95\linewidth}
      \notefont{
        \bgroup
        \color{arylideyellow}
        \hypersetup{linkcolor=amber}
        \setlength{\tabcolsep}{0.3em}
        #2
        \egroup
      }
      \end{varwidth}
    };
    \begin{pgfonlayer}{background}
      \draw [shade, blur shadow={shadow opacity=15}] (current bounding box.north west) rectangle (current bounding box.south east);
    \end{pgfonlayer}
  \end{tikzpicture}
}

\definecolor{darkcellborder}{HTML}{634831}
\definecolor{darkcellbg}{HTML}{20160C}

\newcommand{\darkcell}[2][0.9]{
  \begin{tikzpicture}
    \filldraw[line width=1.0pt, fill=darkcellbg, fill opacity=0.5, draw=darkcellborder] (0, 0) rectangle (\linewidth, #1);
    \node[text width=\linewidth, align=center] at (current bounding box.center) {\textbf{#2}};
  \end{tikzpicture}
}

\definecolor{lightcellborder}{HTML}{77543e}
\definecolor{lightcellbg}{HTML}{20160C}

\newcommand{\lightcell}[2][0.9]{
  \begin{tikzpicture}
    \filldraw[line width=1.0pt, fill=lightcellbg, fill opacity=0.25, draw=lightcellborder] (0, 0) rectangle (\linewidth, #1);
    \node[text width=\linewidth, align=center] at (current bounding box.center) {\color{white}#2};
  \end{tikzpicture}
}

% Commands to be used for automation generating printable version
\newcommand{\pagetarget}[2]{\label{#1}\hypertarget{#1}{#2}}
\newcommand{\pagelink}[2]{\hyperlink{#1}{#2}\iftoggle{printable}{ \textmd{(\pageshorthand\,\pageref{#1})}}{}}

% Command for overlay circled text
\definecolor{goblin}{HTML}{3b7c33}
\newcommand\encircle[1]{%
  \tikz[baseline=(X.base)]
  \node (X) [draw=white, shape=circle, inner sep=0, fill=goblin, text=white, blur shadow={shadow blur steps=5}] {\strut \textbf{#1}};%
}

% Background
\AddToHook{shipout/background}{%
  \iftoggle{printable}{}{\put (0in,-\paperheight){\includegraphics[width=\paperwidth,height=\paperheight]{\layout/tausta.png}}}%
  \put (0in,-\paperheight){\includegraphics[width=\paperwidth,height=0.05\paperheight]{\layout/bottom.png}}%
}

\makeindex[columns=3, title=,]

\begin{document}

% !TeX spellcheck = en_US
\thispagestyle{empty}
\begin{tikzpicture}[remember picture, overlay, inner sep=10pt]
  \iftoggle{printable}{}{\node(cover)[anchor=center] at (current page.center) {
    \includegraphics[height=\paperheight, keepaspectratio]{\layout/cover.jpg}
  };}
  \node(title)[minimum width = \paperwidth, anchor=center, yshift=-10em] at (current page.north) {
    \includegraphics[width=0.6\paperwidth]{\layout/cover_title.png}
  };
  \node(subtitle)[anchor=center, yshift=10em] at (current page.south) {
    \includegraphics[width=0.6\paperwidth]{\layout/cover_subtitle.png}
  };
\end{tikzpicture}


\author{Hermanni ``Heegu'' Karppela}
\maketitle

\begin{center}
  \version{}

  \bigbreak

  \intro{}

  \iftoggle{printable}{}{\bigbreak\heegusquote}
\end{center}

\iftoggle{printable}{
  \bigbreak

  \begin{multicols}{2}
  \centering
  \includegraphics[width=0.8\linewidth]{\qr/github.png}\\
  \qrgithub

  \columnbreak

  \includegraphics[width=0.8\linewidth]{\qr/bgg.png}\\
  \qrbgg
  \end{multicols}
}{}

\begin{tikzpicture}[remember picture, overlay]
  \node(cover)[anchor=center, yshift=12em] at (current page.south) {
    \includegraphics[width=1.01\paperwidth, keepaspectratio]{\art/castle_bottom.png}
    \thispagestyle{empty}
  };
\end{tikzpicture}

\clearpage

\begin{multicols*}{2}
\tableofcontents
\vspace*{\fill}
\columnbreak
\vspace*{\fill}
\includegraphics[width=\linewidth]{\art/black_dragon.jpg}
\end{multicols*}

\clearpage

\include{\sections/introduction.tex}

\include{\sections/game_modes.tex}

\include{\sections/components.tex}

\include{\sections/setup.tex}

\include{\sections/round_structure.tex}

\include{\sections/player_turns.tex}

\include{\sections/heroes.tex}

\include{\sections/deckbuilding.tex}

\include{\sections/resources.tex}

\include{\sections/town.tex}

\include{\sections/map_elements.tex}

\include{\sections/units.tex}

\include{\sections/combat.tex}

\iftoggle{printable}{\include{\sections/quote_page.tex}}{}

\include{\sections/ai_rules.tex}

\include{\sections/difficulty.tex}

\include{\sections/trading.tex}

\include{\sections/scenario_end.tex}

\include{\sections/expansion_content.tex}

\include{\sections/all_map_locations.tex}

\iftoggle{printable}{% !TeX spellcheck = en_US
\addsection{Notes}{\spells/cure.png}

\AddToHookNext{shipout/background}{%
  \iftoggle{printable}{}{\put (0in,-\paperheight){\includegraphics[width=\paperwidth,height=\paperheight]{\layout/castle_background.jpg}}}%
  \put (0in,-\paperheight){\includegraphics[width=\paperwidth,height=0.05\paperheight]{\layout/bottom.png}}%
}
}{}

\include{\sections/credits.tex}

\iftoggle{printable}{\include{\sections/index.tex}}{}

% !TeX spellcheck = en_US
\pagestyle{empty}
\iftoggle{printable}{}{
  \begin{tikzpicture}[remember picture, overlay, inner sep=10pt]
    \node(cover)[anchor=center] at (current page.center) {
      \includegraphics[height=\paperheight, keepaspectratio]{\layout/back.jpg}
    };
  \end{tikzpicture}
}


\hommtable{21}{
  \pagetarget{Difficulty Table}{}
  \centering
  \medskip
  \textbf{Field Difficulty Level Table}\\
  \bigskip

  \newcommand{\bronze}[0]{\includesvg[height=12px]{\svgs/bronze.svg}}
  \newcommand{\silver}[0]{\includesvg[height=12px]{\svgs/silver.svg}}
  \newcommand{\golden}[0]{\includesvg[height=12px]{\svgs/golden.svg}}
  \newcommand{\azure}[0]{\includesvg[height=12px]{\svgs/azure.svg}}
  \begin{tabularx}{\linewidth}{p{0.15\linewidth}XXXX} & \darkcell{Easy} & \darkcell{Normal} & \darkcell{Hard} & \darkcell{Impossible} \\
    \darkcell{Level I} & \lightcell{\bronze} & \lightcell{\bronze} & \lightcell{\bronze \bronze} & \lightcell{\bronze \bronze \bronze} \\
    \darkcell{Level II} & \lightcell{\bronze \bronze} & \lightcell{\bronze \bronze} & \lightcell{\bronze \bronze \bronze} & \lightcell{\bronze \bronze \silver} \\
    \darkcell{Level III} & \lightcell{\bronze \silver} & \lightcell{\bronze \bronze \silver} & \lightcell{\bronze \silver \silver} & \lightcell{\silver \silver \silver} \\
    \darkcell{Level IV} & \lightcell{\bronze \bronze \silver} & \lightcell{\bronze \silver \silver} & \lightcell{\silver \silver \silver} & \lightcell{\silver \silver \golden} \\
    \darkcell{Level V} & \lightcell{\bronze \bronze \silver \golden} & \lightcell{\bronze \silver \silver \golden} & \lightcell{\silver \silver \golden \golden} & \lightcell{\silver \golden \golden \golden} \\
    \darkcell{Level VI} & \lightcell{\bronze \bronze \silver \silver \golden} & \lightcell{\bronze \silver \silver \golden \golden} & \lightcell{\silver \silver \golden \golden \golden} & \lightcell{\silver \golden \golden \golden \golden} \\
    \darkcell{Level VII} & \lightcell{\azure} & \lightcell{\azure \azure} & \lightcell{\golden \azure \azure} & \lightcell{\golden \golden \azure \azure} \\
  \end{tabularx}
}

\bigskip

\hommtable{14}{
  \centering
  \textbf{\pagetarget{Trade Table}{Trade Table}}\\
  \bigskip

  \begin{tabularx}{\linewidth}{XXXX}
    \darkcell{\color{white}Sells/gets} & \darkcell{...to purchase \includesvg[height=10px]{\svgs/gold.svg}} & \darkcell{...to purchase \includesvg[height=10px]{\svgs/valuablegreater.svg}} & \darkcell{...to purchase \includesvg[height=10px]{\svgs/building_materials.svg}} \\
    \darkcell{I am selling \includesvg[height=10px]{\svgs/gold.svg}...} & \lightcell{--} & \lightcell{6 \includesvg[height=10px]{\svgs/gold.svg} \includesvg[height=0.5\baselineskip]{\svgs/arrow_right_gray.svg} 1 \includesvg[height=10px]{\svgs/valuablegreater.svg}} & \lightcell{2 \includesvg[height=10px]{\svgs/gold.svg} \includesvg[height=0.5\baselineskip]{\svgs/arrow_right_gray.svg} 1 \includesvg[height=10px]{\svgs/building_materials.svg}} \\
    \darkcell{I am selling \includesvg[height=10px]{\svgs/valuablegreater.svg}...} & \lightcell{1 \includesvg[height=10px]{\svgs/valuablegreater.svg} \includesvg[height=0.5\baselineskip]{\svgs/arrow_right_gray.svg} 3 \includesvg[height=10px]{\svgs/gold.svg}} & \lightcell{--} & \lightcell{1 \includesvg[height=10px]{\svgs/valuablegreater.svg} \includesvg[height=0.5\baselineskip]{\svgs/arrow_right_gray.svg} 2 \includesvg[height=10px]{\svgs/building_materials.svg}} \\
    \darkcell{I am selling \includesvg[height=10px]{\svgs/building_materials.svg}...} & \lightcell{1 \includesvg[height=10px]{\svgs/building_materials.svg} \includesvg[height=0.5\baselineskip]{\svgs/arrow_right_gray.svg} 1 \includesvg[height=10px]{\svgs/gold.svg}} & \lightcell{3 \includesvg[height=10px]{\svgs/building_materials.svg} \includesvg[height=0.5\baselineskip]{\svgs/arrow_right_gray.svg} 1 \includesvg[height=10px]{\svgs/valuablegreater.svg}} & \lightcell{--} \\
  \end{tabularx}
}


\end{document}

\begin{document}

% !TeX spellcheck = en_US
\thispagestyle{empty}
\begin{tikzpicture}[remember picture, overlay, inner sep=10pt]
  \iftoggle{printable}{}{\node(cover)[anchor=center] at (current page.center) {
    \includegraphics[height=\paperheight, keepaspectratio]{\layout/cover.jpg}
  };}
  \node(title)[minimum width = \paperwidth, anchor=center, yshift=-10em] at (current page.north) {
    \includegraphics[width=0.6\paperwidth]{\layout/cover_title.png}
  };
  \node(subtitle)[anchor=center, yshift=10em] at (current page.south) {
    \includegraphics[width=0.6\paperwidth]{\layout/cover_subtitle.png}
  };
\end{tikzpicture}


\iftoggle{printable}{
  \newgeometry{
    twoside,
    top=2cm,
    bottom=3cm,
    left=2.5cm,
    right=1.5cm,
    marginparwidth=1.75cm,
    footskip=2.05cm,
  }
}{}


\author{Hermanni ``Heegu'' Karppela}
\maketitle

\begin{center}
  \iftoggle{githubbuild}{
    \getenv[\githubsha]{GITHUB_SHA}
    \versionwarning{} \href{\repourl}{\StrLeft{\githubsha}{7}}.
  }{
    \versionlabel{} \input{.version}
  }

  \bigbreak

  \intro{}

  \iftoggle{printable}{}{\bigbreak\heegusquote}
\end{center}

\iftoggle{printable}{
  \bigbreak

  \begin{multicols}{2}
  \centering
  \includegraphics[width=0.8\linewidth]{\qr/github.png}\\
  \qrgithub

  \columnbreak

  \includegraphics[width=0.8\linewidth]{\qr/bgg.png}\\
  \qrbgg
  \end{multicols}
}{}

\begin{tikzpicture}[remember picture, overlay]
  \node(cover)[anchor=center, yshift=12em] at (current page.south) {
    \includegraphics[width=1.01\paperwidth, keepaspectratio]{\art/castle_bottom.png}
    \thispagestyle{empty}
  };
\end{tikzpicture}

\clearpage

\begin{multicols*}{2}
\tableofcontents
\vspace*{\fill}
\columnbreak
\vspace*{\fill}
\includegraphics[width=\linewidth]{\art/black_dragon.jpg}
\end{multicols*}

\clearpage

\include{\sections/introduction.tex}

\include{\sections/game_modes.tex}

\include{\sections/components.tex}

\include{\sections/setup.tex}

\include{\sections/round_structure.tex}

\include{\sections/player_turns.tex}

\include{\sections/heroes.tex}

\include{\sections/deckbuilding.tex}

\include{\sections/resources.tex}

\include{\sections/town.tex}

\include{\sections/map_elements.tex}

\include{\sections/units.tex}

\include{\sections/combat.tex}

\iftoggle{printable}{\include{\sections/quote_page.tex}}{}

\include{\sections/ai_rules.tex}

\include{\sections/difficulty.tex}

\include{\sections/trading.tex}

\include{\sections/scenario_end.tex}

\include{\sections/expansion_content.tex}

\include{\sections/all_map_locations.tex}

\iftoggle{printable}{\include{\sections/index.tex}}{}

\include{\sections/credits.tex}

\iftoggle{printable}{% !TeX spellcheck = en_US
\addsection{Notes}{\spells/cure.png}

\AddToHookNext{shipout/background}{%
  \iftoggle{printable}{}{\put (0in,-\paperheight){\includegraphics[width=\paperwidth,height=\paperheight]{\layout/castle_background.jpg}}}%
  \put (0in,-\paperheight){\includegraphics[width=\paperwidth,height=0.05\paperheight]{\layout/bottom.png}}%
}
}{}

% !TeX spellcheck = en_US
\pagestyle{empty}
\iftoggle{printable}{}{
  \begin{tikzpicture}[remember picture, overlay, inner sep=10pt]
    \node(cover)[anchor=center] at (current page.center) {
      \includegraphics[height=\paperheight, keepaspectratio]{\layout/back.jpg}
    };
  \end{tikzpicture}
}


\hommtable{21}{
  \pagetarget{Difficulty Table}{}
  \centering
  \medskip
  \textbf{Field Difficulty Level Table}\\
  \bigskip

  \newcommand{\bronze}[0]{\includesvg[height=12px]{\svgs/bronze.svg}}
  \newcommand{\silver}[0]{\includesvg[height=12px]{\svgs/silver.svg}}
  \newcommand{\golden}[0]{\includesvg[height=12px]{\svgs/golden.svg}}
  \newcommand{\azure}[0]{\includesvg[height=12px]{\svgs/azure.svg}}
  \begin{tabularx}{\linewidth}{p{0.15\linewidth}XXXX} & \darkcell{Easy} & \darkcell{Normal} & \darkcell{Hard} & \darkcell{Impossible} \\
    \darkcell{Level I} & \lightcell{\bronze} & \lightcell{\bronze} & \lightcell{\bronze \bronze} & \lightcell{\bronze \bronze \bronze} \\
    \darkcell{Level II} & \lightcell{\bronze \bronze} & \lightcell{\bronze \bronze} & \lightcell{\bronze \bronze \bronze} & \lightcell{\bronze \bronze \silver} \\
    \darkcell{Level III} & \lightcell{\bronze \silver} & \lightcell{\bronze \bronze \silver} & \lightcell{\bronze \silver \silver} & \lightcell{\silver \silver \silver} \\
    \darkcell{Level IV} & \lightcell{\bronze \bronze \silver} & \lightcell{\bronze \silver \silver} & \lightcell{\silver \silver \silver} & \lightcell{\silver \silver \golden} \\
    \darkcell{Level V} & \lightcell{\bronze \bronze \silver \golden} & \lightcell{\bronze \silver \silver \golden} & \lightcell{\silver \silver \golden \golden} & \lightcell{\silver \golden \golden \golden} \\
    \darkcell{Level VI} & \lightcell{\bronze \bronze \silver \silver \golden} & \lightcell{\bronze \silver \silver \golden \golden} & \lightcell{\silver \silver \golden \golden \golden} & \lightcell{\silver \golden \golden \golden \golden} \\
    \darkcell{Level VII} & \lightcell{\azure} & \lightcell{\azure \azure} & \lightcell{\golden \azure \azure} & \lightcell{\golden \golden \azure \azure} \\
  \end{tabularx}
}

\bigskip

\hommtable{14}{
  \centering
  \textbf{\pagetarget{Trade Table}{Trade Table}}\\
  \bigskip

  \begin{tabularx}{\linewidth}{XXXX}
    \darkcell{\color{white}Sells/gets} & \darkcell{...to purchase \includesvg[height=10px]{\svgs/gold.svg}} & \darkcell{...to purchase \includesvg[height=10px]{\svgs/valuablegreater.svg}} & \darkcell{...to purchase \includesvg[height=10px]{\svgs/building_materials.svg}} \\
    \darkcell{I am selling \includesvg[height=10px]{\svgs/gold.svg}...} & \lightcell{--} & \lightcell{6 \includesvg[height=10px]{\svgs/gold.svg} \includesvg[height=0.5\baselineskip]{\svgs/arrow_right_gray.svg} 1 \includesvg[height=10px]{\svgs/valuablegreater.svg}} & \lightcell{2 \includesvg[height=10px]{\svgs/gold.svg} \includesvg[height=0.5\baselineskip]{\svgs/arrow_right_gray.svg} 1 \includesvg[height=10px]{\svgs/building_materials.svg}} \\
    \darkcell{I am selling \includesvg[height=10px]{\svgs/valuablegreater.svg}...} & \lightcell{1 \includesvg[height=10px]{\svgs/valuablegreater.svg} \includesvg[height=0.5\baselineskip]{\svgs/arrow_right_gray.svg} 3 \includesvg[height=10px]{\svgs/gold.svg}} & \lightcell{--} & \lightcell{1 \includesvg[height=10px]{\svgs/valuablegreater.svg} \includesvg[height=0.5\baselineskip]{\svgs/arrow_right_gray.svg} 2 \includesvg[height=10px]{\svgs/building_materials.svg}} \\
    \darkcell{I am selling \includesvg[height=10px]{\svgs/building_materials.svg}...} & \lightcell{1 \includesvg[height=10px]{\svgs/building_materials.svg} \includesvg[height=0.5\baselineskip]{\svgs/arrow_right_gray.svg} 1 \includesvg[height=10px]{\svgs/gold.svg}} & \lightcell{3 \includesvg[height=10px]{\svgs/building_materials.svg} \includesvg[height=0.5\baselineskip]{\svgs/arrow_right_gray.svg} 1 \includesvg[height=10px]{\svgs/valuablegreater.svg}} & \lightcell{--} \\
  \end{tabularx}
}


\end{document}
