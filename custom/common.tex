% !TeX spellcheck = en_US
\documentclass[12pt]{article}

% Language setting
\usepackage[T1]{fontenc}
\usepackage[english]{babel}

\def\sections{sections}
\def\lang_header_adjustment{0px}

\newcommand{\pageshorthand}{p.}

\title{\includegraphics[width=6cm]{\images/title.png}\\The Rewrite}

% Set page size and margins
\usepackage[
  a4paper,
  top=2cm,
  bottom=3cm,
  left=2cm,
  right=2cm,
  marginparwidth=1.75cm,
  footskip=2.05cm
]{geometry}

% Useful packages
\usepackage[export]{adjustbox}
\usepackage{amsmath}
\usepackage{array}
\usepackage{caption}
\usepackage[strict]{changepage}
\usepackage{enumitem}
\usepackage{etoolbox}
\usepackage{float}
\usepackage{fullwidth}
\usepackage{graphicx, trimclip}
\usepackage[colorlinks=true, allcolors=blue]{hyperref}
\usepackage{hyperref}
\usepackage[noautomatic, nonewpage]{imakeidx}
\usepackage{multicol}
\usepackage[super]{nth}
\usepackage{outlines}
\usepackage{paracol}
\usepackage[section]{placeins}
\usepackage{setspace}
\usepackage{stfloats}
\usepackage{subcaption}
\usepackage[usetransparent=false]{svg}
\usepackage{tabularx}
\usepackage[subfigure]{tocloft}
\usepackage{tikz}
\usepackage{titlesec}
\usepackage{verbatim}
\usepackage{varwidth}
\usepackage{wrapfig}
\usepackage{soul}
\usepackage[most]{tcolorbox}
\newtcolorbox{scaledfigure}[1][]{height fill, space to=\myspace,#1}
\hypersetup{
  colorlinks=true,
  linkcolor=goldenbrown,
  filecolor=magenta,
  urlcolor=cyan,
  pdftitle={Heroes of Might \& Magic III Rule Book},
  pdfpagemode=UseNone,
}
% Set the default spacing between paragraphs. Remove indentation.
\usepackage[skip=6pt, indent=0pt]{parskip}
\setstretch{1}

% Add dots to the table of contents
\renewcommand{\cftsecleader}{\cftdotfill{\cftsecdotsep}}
\renewcommand\cftsecdotsep{\cftdot}
\renewcommand\cftsubsecdotsep{\cftdot}

\captionsetup[figure]{labelformat=empty}
\captionsetup[subfigure]{labelformat=empty, singlelinecheck=off, justification=centering}
\usetikzlibrary{shadows, shadows.blur, calc, backgrounds}

\setlength{\columnsep}{1cm}
\newtoggle{printable}

% Variables
\def\_assets{assets}

\def\art{\_assets/art}
\def\cards{\_assets/cards}
\def\examples{\_assets/examples}
\def\images{\_assets/images}
\def\layout{\_assets/layout}
\def\map_locations{\_assets/map-locations}
\def\skills{\_assets/skills}
\def\spells{\_assets/spells}
\def\svgs{\_assets/glyphs}
\def\notes_svgs{\svgs/for-notes}
\def\tables{\_assets/tables}
\def\qr{\_assets/qr-codes}

\def\custom{custom}
\def\scenarios{\custom/scenarios}
\def\rules{\custom/rules}

\sodef\sotitle{}{0.2em}{0.6em}{1.2em}

\def\gold{\includesvg[height=10px]{\svgs/gold.svg}}
\def\valuables{\includesvg[height=10px]{\svgs/valuablegreater.svg}}
\def\materials{\includesvg[height=10px]{\svgs/building_materials.svg}}
\def\spellpower{\includesvg[height=10px]{\svgs/spellpower.svg}}
\def\pay{\includesvg[height=10px]{\svgs/pay_v2.svg}}
\def\expert{\includesvg[height=10px]{\svgs/expert.svg}}
\def\moralepos{\includesvg[height=10px]{\svgs/morale_positive.svg}}
\def\moraleneg{\includesvg[height=10px]{\svgs/morale_negative.svg}}
\def\paralysis{\includesvg[height=10px]{\svgs/paralysis.svg}}

\renewcommand{\labelitemi}{\includegraphics[width=0.7em, valign=c]{\layout/listdot.png}}

% Colors
\definecolor{amber}{rgb}{1.0, 0.49, 0.0}
\definecolor{antiquewhite}{rgb}{0.98, 0.92, 0.84}
\definecolor{arylideyellow}{rgb}{0.96, 0.89, 0.58}
\definecolor{cadmiumgreen}{rgb}{0.0, 0.42, 0.24}
\definecolor{darkcandyapplered}{rgb}{0.64, 0.0, 0.0}
\definecolor{goldenbrown}{rgb}{0.6, 0.4, 0.08}

% Command to frame images
\newcommand\framedimage[2][]{%
  \begin{tikzpicture}
    \draw (0, 0) node[inner sep=0] {\makebox[#1][c]{\includegraphics[width=#1]{#2}}};
    \draw [bordermidyellow, thick] ([xshift=+1pt, yshift=-1pt] current bounding box.north west) rectangle ([xshift=-1pt, yshift=1pt] current bounding box.south east);
    \draw [borderoutyellow, thick] (current bounding box.north west) rectangle (current bounding box.south east);
    \draw [borderinyellow, thick] ([xshift=+3pt, yshift=-3pt] current bounding box.north west) rectangle ([xshift=-3pt, yshift=3pt] current bounding box.south east);
  \end{tikzpicture}}
% End of drop frame definition

\titleformat{\section}
{\huge}
{\filright
\footnotesize
\enspace SECTION \thesection\enspace}
{8pt}
{\Huge\bfseries\filcenter\uppercase}
%Create section heading with graphics. Argument one is heading name, argument two is picture to use on the left.
\providecommand{\sectionheadertext}[1]{
  \fontfamily{ptm}\selectfont{
    \color{antiquewhite} \section*{\MakeUppercase{#1}}
  }
}
\newcommand{\addsection}[2]{
  \vspace*{-5em}
  \hspace*{-1em}
  \makebox[0pt][l]{
  \raisebox{-\totalheight}[0pt][7pt]{
    \begin{tikzpicture}
      \draw (0, 0) node[inner sep=0] {\includegraphics[width=\linewidth, height=0.2\linewidth]{\layout/section_heading.jpg}};
      \draw (-6.2, 0) node {\includegraphics[width=0.125\textwidth]{#2}};
    \end{tikzpicture}
    }
  }
  \begin{fullwidth}[leftmargin=0.16\textwidth]
    \begin{center}
      \vspace*{\lang_header_adjustment}
      \sectionheadertext{#1}
      \cleardoublepage\phantomsection\addcontentsline{toc}{section}{\protect\numberline{}#1}
    \end{center}
  \end{fullwidth}
  \vspace{1.75em}
}
\newcommand{\addscenariosection}[3]{
  \vspace*{-5em}
  \hspace*{-1em}
  \makebox[0pt][l]{
  \raisebox{-\totalheight}[0pt][7pt]{
    \begin{tikzpicture}
      \draw (0, 0) node[inner sep=0, yscale=1.3] {\includegraphics[width=\linewidth, height=0.2\linewidth]{\layout/section_heading.jpg}};
      \draw (-6.2, 0) node {\includegraphics[width=0.125\textwidth]{#3}};
    \end{tikzpicture}
    }
  }
  \begin{fullwidth}[leftmargin=0.16\textwidth]
    \begin{center}
      \vspace{-22pt}
      \sectionheadertext{\small{\sotitle{#1}}}
      \vspace{-12pt}
      \sectionheadertext{#2}
      \cleardoublepage\phantomsection\addcontentsline{toc}{section}{\protect\numberline{}#2}
    \end{center}
  \end{fullwidth}
  \vspace{1.75em}
}
%End of create section heading.

\newcommand\picdims[4][]{%
  \setbox0=\hbox{\includegraphics[#1]{#4}}%
  \clipbox{.5\dimexpr\wd0-#2\relax{} %
    .5\dimexpr\ht0-#3\relax{} %
    .5\dimexpr\wd0-#2\relax{} %
    .5\dimexpr\ht0-#3\relax}{\includegraphics[#1]{#4}}}

\tikzset{
  thick/.style=      {line width=1.3pt},
  very thick/.style= {line width=1.7pt},
  ultra thick/.style={line width=2.2pt}
}

\definecolor{borderoutyellow}{HTML}{DBCA86}
\definecolor{borderinyellow}{HTML}{B09E69}
\definecolor{bordermidyellow}{HTML}{6f6749}
% Create note box
\providecommand{\notefont}[0]{\fontfamily{ptm}\selectfont}
\newcommand{\note}[2]{
  \begin{tikzpicture}
    \draw (0, 0) node[inner sep=0] {\makebox[\linewidth][c]{\picdims[width=\linewidth]{\linewidth}{#1\baselineskip}{\layout/table-background.jpg}}};
    \draw [borderoutyellow, very thick] (current bounding box.north west) rectangle (current bounding box.south east);
    \draw [borderinyellow, thick] ([xshift=+2.8pt, yshift=-2.8pt] current bounding box.north west) rectangle ([xshift=-2.8pt, yshift=2.8pt] current bounding box.south east);
    \node at (current bounding box.center) {
      \begin{varwidth}{0.85\linewidth}
      \notefont{
        \color{arylideyellow}
        \hypersetup{linkcolor=amber}
        #2
        \hypersetup{linkcolor=goldenbrown}
      }
      \end{varwidth}
    };
    \begin{pgfonlayer}{background}
      \draw [shade, blur shadow={shadow opacity=15}] (current bounding box.north west) rectangle (current bounding box.south east);
    \end{pgfonlayer}
  \end{tikzpicture}
}

\newcommand{\hommtable}[2]{
  \begin{tikzpicture}
    \draw (0, 0) node[inner sep=0] {\makebox[\linewidth][c]{\picdims[width=\linewidth]{\linewidth}{#1\baselineskip}{\layout/table-background.jpg}}};
    \draw [bordermidyellow, thick] ([xshift=+1pt, yshift=-1pt] current bounding box.north west) rectangle ([xshift=-1pt, yshift=1pt] current bounding box.south east);
    \draw [borderoutyellow, thick] (current bounding box.north west) rectangle (current bounding box.south east);
    \draw [borderinyellow, thick] ([xshift=+3pt, yshift=-3pt] current bounding box.north west) rectangle ([xshift=-3pt, yshift=3pt] current bounding box.south east);
    \node at (current bounding box.center) {
      \begin{varwidth}{0.95\linewidth}
      \notefont{
        \bgroup
        \color{arylideyellow}
        \hypersetup{linkcolor=amber}
        \setlength{\tabcolsep}{0.3em}
        #2
        \egroup
      }
      \end{varwidth}
    };
    \begin{pgfonlayer}{background}
      \draw [shade, blur shadow={shadow opacity=15}] (current bounding box.north west) rectangle (current bounding box.south east);
    \end{pgfonlayer}
  \end{tikzpicture}
}

\definecolor{darkcellborder}{HTML}{634831}
\newcommand{\darkcell}[2][0.9]{
  \begin{tikzpicture}
    \filldraw[line width=1.0pt, fill=black, fill opacity=0.5, draw=darkcellborder] (0, 0) rectangle (\linewidth, #1);
    \node[text width=\linewidth, align=center] at (current bounding box.center) {\textbf{#2}};
  \end{tikzpicture}
}

\definecolor{lightcellborder}{HTML}{77543e}
\newcommand{\lightcell}[2][0.9]{
  \begin{tikzpicture}
    \filldraw[line width=1.0pt, fill=black, fill opacity=0.2, draw=lightcellborder] (0, 0) rectangle (\linewidth, #1);
    \node[text width=\linewidth, align=center] at (current bounding box.center) {\color{white}#2};
  \end{tikzpicture}
}

% Command for overlay circled text
\definecolor{goblin}{HTML}{3b7c33}
\newcommand\encircle[1]{%
  \tikz[baseline=(X.base)]
  \node (X) [draw=white, shape=circle, inner sep=0, fill=goblin, text=white, blur shadow={shadow blur steps=5}] {\strut \textbf{#1}};%
}

